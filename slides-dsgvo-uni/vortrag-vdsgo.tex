\subsection{Kontext}

\frame{
	{\Large Herzlich Willkommen }
	\begin{itemize}
		\item Mobilfunk aus
		\item Der Vortrag wird aufgenommen: https://podcast.chaostreff-siegen.de/
		\item Der Vortrag ist Public Domain
		\item Der Workshop wurde schon ein mal im Hackspace Siegen gehalten: https://hasi.it
		\item Was ist Chaos Siegen? https://chaos-siegen.de
	\end{itemize}
}

\subsection{Vortrag VDSGO}

\frame{
	{\Large Datenschutz-Grundverordnung}
	\begin{itemize}
		\item IANAL (I am not a lawyer)
		\item Titel: Verordnung des Europäischen Parlaments und des Rates zum Schutz natürlicher Personen bei der Verarbeitung personenbezogener Daten, zum freien Datenverkehr und zur Aufhebung der Richtlinie 95/46/EG
		\item Link: \url{http://eur-lex.europa.eu/legal-content/DE/TXT/HTML/?uri=CELEX:02016R0679-20160504&from=EN}
		\item Was jetzt? 
	\end{itemize}
}

\frame{
	{\Large Hype}
	\begin{itemize}
		\item Aufruhr um DSGVO in BRD recht witzig
		\item BDSG, TKG, TMG ähneln der DSVGO sehr.
		\item DSVGO geltendes Recht seit 24. Mai 2016
		\item Anzuwendendes Recht seit 25. Mai 2018
		\item Es gibt kleine Veränderungen:
		\begin{itemize}
			\item Strafen sind jetzt höher\dots
			\item \dots und kleinere Änderungen
		\end{itemize}
		 \item Das Ding ist 62 Seiten lang. Einfach mal lesen.
	\end{itemize}
}

\frame{
	{\Large Beispiel \enquote{Verarbeitung, Art. 4 DSGVO}}
	\enquote{Im Sinne dieser Verordnung bezeichnet der Ausdruck: [\dots] \enquote{Verarbeitung} jeden mit oder ohne Hilfe automatisierter Verfahren ausgeführten Vorgang oder jede solche Vorgangsreihe im Zusammenhang mit personenbezogenen Daten wie das Erheben, das Erfassen, die Organisation, das Ordnen, die Speicherung, die Anpassung oder Veränderung, das Auslesen, das Abfragen, die Verwendung, die Offenlegung durch Übermittlung, Verbreitung oder eine andere Form der Bereitstellung, den Abgleich oder die Verknüpfung, die Einschränkung, das Löschen oder die Vernichtung;}
	\begin{itemize}
		\item Wie ist das in der Praxis?
	\end{itemize}
}

\frame{
	{\Large Beispiele aus der Praxis zu \enquote{Verarbeitung, Art. 4 DSGVO}}
	\begin{itemize}
		\item Webseiten binden Javascripte von JSdeliver, Google, Bing, Yahoo usw. ein ohne vorher zu
		informieren.
		\item Wenn über das Laden Inhalte Dritter und das Übermitteln der Daten an Dritte informiert wird, ist es schon zu spät (Bestellung bei Online Shops)
		\item Cookie-Hinweis wird angezeigt während im Hintergrund die Cookies bereits geladen sind (Google Analytics trackt die Mausbewegung, die notwendig ist um den Hinweis weg zu klicken).
	\end{itemize}
}

\frame{
	\begin{figure}
	\centering
	\includegraphics[width=0.5\textwidth]{../pics/we-are-doomed.jpeg}
	\caption{comicvintage.tumblr.com | Digital Millennium Copyright Act }
	\end{figure}
}

\frame{
	{\Large Themenwechsel}
	\begin{itemize}
		\item Wenn dein Verein in einem Dachverband ist, dann ist er bereits informiert worden und knappe Literatur ist empfohlen worden (12 Seiten lang).
		\item Wenn nicht, hier ist die Literaturempfehlung: Unabhängiges Landeszentrum für Datenschutz Schleswig-Holstein: https://www.datenschutzzentrum.de/uploads/praxisreihe/PraxisReihe-1-Datenschutz-bei-Vereinen.pdf 
		\item Auch im git.
	\end{itemize}
}

\frame{
	{\Large Der Teil für Vereine: Grob, 1}
	\begin{itemize}
		\item Anzuwenden auf: Alle Vereinsarten. Ohne Einschränkung.
		\item Personenbezug: Personenbezogene Daten
		\item Verkackt: Tatbestand einer Ordnungswidrigkeit
		\item Verknackt: Der Verein, aber auch Vorstand oder Verkacker möglich
	\end{itemize}
}


\frame{
	{\Large Der Teil für Vereine: Datensammeln, 2}
	\begin{itemize}
		\item Datensammeln bei: Aufnahmebogen, Webformular
		\item Direktwerbung, Newsletter? Einwilligung notwendig
		\item Erste Direktwerbung: Information zum Widerspruchsrecht beilegen
		\item Fotografien, Fotos auf der Internetseite, Vereinsmagazin? Einwilligung des Abgebildeten
		\item Einfach mal ne E-Mail an die Mitglieder: Grundrechte und Grundfreiheit aller Betroffener abwägen
	\end{itemize}
Lösung: Bei der Anmeldung die Verarbeitungszwecke auflisten und per Unterschrift zustimmen lassen. Das Mitglied kann aber einzelnen Zwecken widersprechen dürfen.
}

\frame{
	{\Large Der Teil für Vereine: Satzung, 3}
	\begin{itemize}
		\item Grundsätzlich nicht Daten sammeln
		\item Außer Vereinszwecke können sonst nicht erfüllt werden: In Satzung erlauben lassen.
		\item Aber für Standardkram Erlaubnis entbehrlich: Vorstand im Vereinsregister bekannt geben
	\end{itemize}
}

\frame{
	{\Large Der Teil für Vereine: Rechte der Mitglieder (natürliche Person), 4}
	\begin{itemize}
		\item Informationspflichten bei der Erhebung von personenbezogenen
		Daten
		\item auf Auskunft
		\item Berichtigung
		\item Löschung
		\item Einschränkung der Verarbeitung
		\item Mitteilungspflicht in Bezug auf die Berichtigung, Löschung oder Einschränkung der Verarbeitung
		\item Datenübertragbarkeit
		\item Widerspruchsrecht
		\item Recht, nicht einer ausschließlich auf einer automatisierten Verarbeitung – einschließlich Profiling – beruhenden Entscheidung unterworfen zu werden, die der betroffenen Person gegenüber rechtliche Wirkung entfaltet
		oder sie in ähnlicher Weise erheblich beeinträchtigt
	\end{itemize}
}

\frame{
	{\Large Der Teil für Vereine: Verein, 5}
	\begin{itemize}
		\item Anträge innerhalb eines Monats bearbeiten
		\item Nichtmitgliederdaten verarbeitet? Die haben auch Rechte
		\item Flut von Anträge? Verlängerung auf 2 Monate
	\end{itemize} 
}

\frame{
	{\Large Der Teil für Vereine: Datenweitergabe, 6}
	\begin{itemize}
		\item Mitglieder wollen sich gegen der Vorstand organisieren und brauchen dazu Daten über die anderen Mitglieder? Treuhänder (Berufsgeheimnisträger) auswählen
		\item Mitglieder wollen eine Fahrgemeinschaft organisieren? Nein, aber Rückfragen und Daten des ursprünglichen Frages geben
		\item Dienstleister? Prüfen, ob der die DSVGO einhält
	\end{itemize} 
}

\frame{
	{\Large Der Teil für Vereine: Bürokratie-Foo, 7}
	\begin{itemize}
		\item Verein muss Verzeichnis von Verarbeitungstätigkeiten führen
		\item Datenschutzbeauftragter? Nein, wenn nur neun Personen ständig mit der automatisierten Verarbeitung
		personenbezogener Daten beschäftigt sind.
		\item Datenschutzbeauftragter? Ja, immer wenn besondere Kernkategorien Zweck des Vereins sind: Gesundheit, Krankheit, Kinder, Jugendliche
		\item Technisch-organisatorische Anforderungen: angemessenes Schutzniveau
	\end{itemize} 
}

\frame{
	{\Large Sapere aude}
	\begin{itemize}
		\item Strikte EU-Datenschutzvorgaben macht? Mark Zuckerbergs verspricht, sich weltweit daran zu halten? Facebook ziehz 1,5 Milliarden User von Irland in die USA um: https://www.theguardian.com/technology/2018/apr/19/facebook-moves-15bn-users-out-of-reach-of-new-european-privacy-lawVerein muss Verzeichnis von Verarbeitungstätigkeiten führen
		\item Lösung für die Geldknappheit der Kommunen; Der Städte- und Gemeindebund empfiehlt: Auch die Kommunen sollten mit den Daten der Bürger Geld verdienen. Dafür schlägt der Geschäftsführer ein Konzessionsmodell vor. \enquote{
		Auch die Städte und Gemeinden müssen sich noch mehr klar machen, dass Daten das Öl des 21. Jahrhunderts sind und sich damit wichtige Einnahmen erzielen lassen}	
	\end{itemize} 
}

\frame{
	{\Large Sapere aude, 1}
	Strikte EU-Datenschutzvorgaben macht? Mark Zuckerbergs verspricht, sich weltweit daran zu halten? Facebook ziehz 1,5 Milliarden User von Irland in die USA um: https://www.theguardian.com/technology/2018/apr/19/facebook-moves-15bn-users-out-of-reach-of-new-european-privacy-lawVerein muss Verzeichnis von Verarbeitungstätigkeiten führen
}

\frame{
	{\Large Sapere aude, 2}
	Lösung für die Geldknappheit der Kommunen; Der Städte- und Gemeindebund empfiehlt: Auch die Kommunen sollten mit den Daten der Bürger Geld verdienen. Dafür schlägt der Geschäftsführer ein Konzessionsmodell vor. \enquote{
			Auch die Städte und Gemeinden müssen sich noch mehr klar machen, dass Daten das Öl des 21. Jahrhunderts sind und sich damit wichtige Einnahmen erzielen lassen}	
}