\subsection{Vortrag}

\frame{
	{\Large Datenschutz }
	\begin{itemize}
		\item Schutz der Menschen, deren Daten erhoben und verwendet werden
		\item Schutz vor Überwachung durch den Staat
		\item Schutz vor Ausbeutung durch Privatunternehmen
	\end{itemize}
}

\frame{
	{\Large Informationelle Selbstbestimmung}
	\begin{itemize}
		\item Grundsatz: Jeder Mensch kann selber bestimmen, wem er welchen Daten bekannt gibt
		\item Personenbezogene Daten: Name, Anschrift, Familienstand
		\item Sachliche Angaben: Eigentumsverhältnisse
	\end{itemize}
}

\frame{
	{\Large Kritiker behaupten}
	\begin{itemize}
		\item Datenschutz ist alt
		\item Bremst technischen und wirtschaftlichen Fortschritt
		\item Kostet Geld
	\end{itemize}
}

\frame{
	{\Large Befürworter behaupten}
	\begin{itemize}
		\item Freiheit
		\item Demokratie
		\item Schutz vor Unterdrückung
		\item Schutz vor Kontrolle
	\end{itemize}
}

\frame{
	{\Large Elektronischer Umgang mit den Daten\dots }
	\dots muss geregelt werden um Missbrauch vorzubeugen
}

\frame{
	{\Large Nationale Bestimmungen\dots}
	\dots lassen sich im Internet kaum durchsetzen
}

\frame{
	{\Large Informationelle Selbstbestimmung \dots}
	\dots ist Eigenverantwortung.
}
